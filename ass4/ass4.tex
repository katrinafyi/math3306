% vim: spelllang=en_au
\documentclass[a4paper]{article}

\usepackage{geometry}
\usepackage{amsmath}
% \usepackage{blkarray}
\usepackage{enumerate}
\usepackage{amssymb}
\usepackage{amsthm}
\usepackage{hyperref}
\usepackage{minted}
% \usepackage[symbol]{footmisc}
\usepackage{tikz}
\usetikzlibrary{automata}
\usetikzlibrary{positioning,arrows.meta,calc}
\usetikzlibrary{angles,intersections,quotes,arrows.meta}

\usepackage[linewidth=0.5pt]{mdframed}

\DeclareMathOperator{\Diag}{Diag}
\DeclareMathOperator{\Th}{Th}

\DeclareMathOperator{\lcm}{lcm}

\author{Kait Lam \\ \small \texttt{45294583} \\ \small {T02}}
\title{\textsc{Math3306} --- Assignment 4 (Submission)}
\date{29 October 2024 (extension)}

\def\dotminus{\mathbin{\ooalign{\hss\raise1ex\hbox{.}\hss\cr
  \mathsurround=0pt$-$}}}

\begin{document}

\maketitle


\section*{Question 1}
% \subsection*{Question 1(a)}
\begin{center}
  \textit{This question asks which elements of $(\mathbb N, +)$ and $(\mathbb Z, +)$ are (0-)definable.}
\end{center}
In $(\mathbb N, +)$, we claim that every element is definable.
First, we define a comparison operator by
$x \le y \equiv (\exists z.~ x + z = y)$.
This can be thought of as a syntactic abbreviation if desired.
Now we make an inductive argument.
Starting with zero, we define 0 as the 
least value of the natural numbers,
$\varphi_0(z) \equiv (\forall y.~ z \le y)$.
This exists and is unique by properties of the natural numbers.
For the inductive case, assume 
there exists $\varphi_n$ defining $n$.
Then, we define $n+1$
as $\varphi_{n+1}(n') \equiv \neg \varphi_n(n')  \wedge (\forall y.\,(\exists x. \,\varphi_n(x)\wedge x \le y) \longrightarrow n' \le y)$.
That is, the least number which is greater than (but not equal to) $n$.
Here, 
$(\exists x. \,\varphi_n(x)\wedge x \le y)$
should be thought of as ``$n \le y$''.
This is a funny trick to give a name to the
value of $n$.

In $(\mathbb Z, +)$, only zero is definable.
Zero is the unique value satisfying 
$\varphi_0(z) \equiv (\forall y. \,y + z = y)$ (additive identity).
To see the remaining numbers are undefinable, we should look at the atomic
formulas expressable with this language;
with only a $+$ function, these formulas have the form
$x_1 + \cdots + x_n = y_1 + \cdots + y_m$.
However, whenever such a formula is satisfied by a valuation $\bar b$,
it is also satisfied by $-(\bar b)$---that is, $\bar b$
with its elements negated.
By induction on formula, this extends to all non-atomic formulas in the language.

This means that if a number satisfies a formula with one free variable,
so does its negation.
Therefore, all non-zero numbers cannot be defined by a formula.
Zero may still be definable (since it is equal to its negation),
and indeed we defined it earlier.



\section*{Question 2}
% \subsection*{Question 1(a)}
\begin{center}
  \textit{If $L$ is a language with only relations and $M$ and $N$ are
  elementarily-equivalent finite $L$-structures, then $M$ and $N$ are isomorphic.}
\end{center}
First, assume $L$ has only finitely many relations.
Let $A_M$ and $A_N$ be the object sets of $M$ and $N$, resp.
Let $|M|$ be the number of elements in $A_M$.
Construct a sentence which is satisfied by a model
iff the model has $|M|$ distinct elements. That is,
\begin{align*}
  \varphi_{1} &= \exists x_1.\,\cdots\exists x_{|M|}.\,
  (x_1 \ne x_2 \wedge \cdots \wedge x_{|M|-1} \ne x_{|M|})\\
              &\quad\wedge (\forall y.\, y = x_1 \vee \cdots\vee y = x_{|M|}).
\end{align*}
This means that for models,
$\mathfrak A \models \varphi_1$ only if $|\mathfrak A| = |M|$.
We should now obtain a sentence which records the information about
how relations in $L$ are satisfied by $M$ and $N$.
Let $\Diag(M)$ be the atomic diagram of $M$;
since $M$ is finite, this is also finite.
Conjoin all sentences in $\Diag(M)$ and make a substitution
of existentially-quantified variables in place
of the constant symbols from $A_M$ (i.e.,
a constant $c_1$ is replaced with $\exists x_1$, though the existential
quantification is shared across all uses of $x_1$).
This formula  (call it $\varphi_2$) now looks like
\begin{align*}
  \varphi_2  &= 
\exists x_1.\,\cdots\exists x_{|M|}.\,
R_1(\cdots) \wedge \neg R_1(\cdots)\wedge R_{|L|}(\cdots) \\
             &\quad\wedge(x_1 \ne x_2 \wedge \cdots \wedge x_{|M|-1} \ne x_{|M|})
\end{align*}
For each of the (finite) relations, it enumerates the (finite) 
Cartesian product of each of the quantified variables,
corresponding to a full specification of the relation on all possible
valuations.

Consider the conjunction $\varphi_1 \wedge \varphi_2$.
This is an elementary sentence.
It is satisfied by $M$ (due to properties of $M$ and its derivation from $\Diag(M)$),
and so it is satisfied by $N$.
Let $y_1, \ldots, y_{|M|}$ be the objects of $N$ witnessing the existential
(and note that $|M| = |N|$).
Construct an isomorphism $h : M \to N$ by $c_i \mapsto y_i$ for each $i$.
To check the relation satisfaction is identical, see that
$R_1^M(c_1, \ldots, c_n)$
or its negation must be in $\Diag(M)$.
By the construction of $\varphi_2$ and elementary equivalence, this is carried over
to the structure $N$ in terms of $y_1, \ldots, y_{|M|}$,
and $h(c_i) = y_i$ for all $i$, so this holds.
Also note $h$ is injective and surjective, due to $|M| = |N|$
and the distinctiveness of $y_1, \ldots, y_{|M|}$ being guaranteed by
$\varphi_1$.

Now for the infinite case where $L$
may have infinitely-many relation.
First, observe that the sentence $\varphi_1$
of the previous part is still satisfied when $L$ is infinite,
as the models themself are finite.
Therefore, $M \models \varphi_1$ and $N \models \varphi_1$,
so $|M| = |N|$.
Now, proceed by contradiction.
Assume that $M$ and $N$ are \textit{not} isomorphic.
That is, for all bijections $h : M \to N$,
there is a relation $R$ such that
$R^M(c_1, \ldots, c_n)$ but not $R^N(y_1, \ldots, y_n)$
where $c_i$ and $y_i$ are related by the bijection ($h(c_i) = y_i$ for all $i$).
[....................]

\begin{mdframed}
  For each bijection $f_i : M \to N$ (of which there are $|M|!$ many),
  obtain a relation $R_i$ which differs at some valuation.
  Construct a language $L_0 \subseteq L$ with only these $\{R_i\}$
  symbols.
  The restrictions of $M$ and $N$ to $L_0$ are still elementarily-equivalent,
  so we invoke the finite case on $L_0$ which tells us that these
  substructures of $M$ and $N$ are isomorphic.
  However, 
\end{mdframed}


\section*{Question 3}
% \subsection*{Question 1(a)}
\begin{center}
  \textit{This question leads us on a journey with existential sentences.}
\end{center}
\subsection*{Question 3(a)}
\begin{center}
  \textit{For existential sentences $\psi$, if $N$ is a substructure of $M$ and $N \models \psi$, then $M \models \psi$.}
\end{center}
This is by induction on the number of existential quantifiers at the front
of the sentence $\psi$.
If there are no quantifiers, then we apply formula-induction on the quantifier-free $\psi$.
Since $N$ is a substructure of $M$, the inclusion map is an embedding,
so
for all terms $t$ and $\bar b$ from $A_N$, $t^N[\bar b] = t^M[\bar b]$.
This alone is sufficient for the $M \models (t_1 = t_2)[\bar b]$ case.
For the $M \models R(t_1, \ldots, t_n)[\bar b]$ case, we also need the property that
a homomorphism respects relations. 
Negation, conjunction, and disjunction hold straighforwardly by induction.

Now for the inductive case (of the main proof), consider a formula of the form $\exists x.\,\psi$. 
By $N \models \exists x.\, \psi[\bar b]$, there exists $a \in A_N$
such that $N \models \psi[\bar b \frac a x]$.
By substructure, we also have that $a \in A_M$ and so
$M \models \psi[\bar b \frac a x]$ and
$M \models \exists x.\,\psi[\bar b]$,
as required.
  

\subsection*{Question 3(b)}
\begin{center}
  \textit{Let $M$ and $N$ be models such that for all existential sentences, $N \models \psi$ implies $M \models \psi$.
  Then, there is a model $L$ such that $L$ is an extension of $N$, and $L$ is elementarily equivalent to $M$.}
\end{center}
We will try to show that $\Diag(N) \cup \Th(M)$ has a model.
In this way, $\Diag(N)$ encodes the object-related properties of $N$
(to  make $L$ extension an extension of $N$),
and $\Th(M)$ ensures $L$ is elementarily equivalent to $M$.

We will do this by compactness.
Take an arbitrary finite subset of $\Diag(N) \cup \Th(M)$.
By tutorial sheet 12's question 1, we can translate occurrences
of constants in this subset (arising from $\Diag(N)$)
with existentially-quantified variables,
e.g. $R(c_1,c_2)$ becomes $\exists x_1.\exists x_2.\, R(x_1, x_2)$,
and these theories are either both consistent or both inconsistent.
These replaced sentences are now existential,
and their construction from within $\Diag(N)$ means
means they were satisfied by $N$, and so they are also satisfied by $M$
due to existential preservation.

Therefore, they are in fact within $\Th(M)$ and so $M$
is a model of this translation of a finite subset of $\Diag(N) \cup \Th(M)$.
By the tutorial's question 1 and compactness,
this tells us there exists a model of $\Diag(N) \cup \Th(M)$.
Call this model $L$. 

Since $L \models \Diag(N)$, by Robinson's lemma,
we have that there is an embedding of $N$ into $L$.
This can be made into an extension by [identifying its objects with the objects of N???????].
Additionally,
since $L \models \Th(M)$, $L$ and $M$ must be elementarily equivalent.
This will be all.

% For each sentence in this subset originating from $\Diag(N)$,
% translate it to a form where its constants are replaced by existential quantification,
% This is now an existential formula, so it is satisfied by $M$ as well as $N$.



\subsection*{Question 3(c)}
\begin{center}
  \textit{Let $T$ be a first-order theory preserved under extensions. Define
    $T_\exists = \{\psi ~|~ \psi\text{ existential and } T \vdash \psi\}$.
    Assume $M \models T_\exists$ and define $\Sigma = \{ \neg \psi ~|~ \psi\text{ existential and }M \nvDash \psi\}$. Then, $T \cup \Sigma$ has a model.
  }
\end{center}
$T_\exists$ is the set of all existential sentences provable by the model $T$,
where $T$ is preserved under extensions.
$\Sigma$ is every existential unsatisfied by $M$,
where $M$ is a model of $T_\exists$.
Preserved under extension: if a structure models $T$, so does
all extensions of it.

Compactness. Take any finite subset of $T \cup \Sigma$.
We have that $M$ satisfies every existential which is proven by $T$. 




\subsection*{Question 3(d)}
\begin{center}
  \textit{Prove $T_\exists \vdash T$.}
\end{center}
\begin{mdframed}
  This question combines (c) and (b)
  to show that the theory $T_\exists$
  (of only positive existential statements)
  is enough to prove
  all of $T$.

  Let $M_\exists$ be a model of $T_\exists$. Let $M_\Sigma$ be a model of $T \cup \Sigma$ as in (c).
  Note that $M_\Sigma \models T$.
  If $M_\Sigma \models \psi$ existential,
  then $\neg \psi \notin \Sigma$ (otherwise inconsistent)
  Thus, $M_\exists \models \psi$ (because it must have come from $T$?).
  Therefore, $M_\Sigma \models \psi$ existential implies $M_\exists \models \psi$.

  By (b), there is a structure $L$ where $L$ is elementarily-equivalent to $M_\exists$
  and $L$ is an extension of $M_\Sigma$.
  Now, $M_\Sigma \models T$ and $T$ is preserved by extensions, so $L \models T$.
  Moreover, since $L$ is elementary-equivalent to $M_\exists$,
  we have $M_\exists \models T$.


\end{mdframed}

\subsection*{Question 3(e)}
\begin{center}
  \textit{A sentence is preserved under extensions iff
  it is logically equivalent to an existential sentence.}
\end{center}
Omitted.



\end{document}

