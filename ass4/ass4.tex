% vim: spelllang=en_au
\documentclass[a4paper]{article}

\usepackage{geometry}
\usepackage{amsmath}
% \usepackage{blkarray}
\usepackage{enumerate}
\usepackage{amssymb}
\usepackage{amsthm}
\usepackage{hyperref}
\usepackage{minted}
% \usepackage[symbol]{footmisc}
\usepackage{tikz}
\usetikzlibrary{automata}
\usetikzlibrary{positioning,arrows.meta,calc}
\usetikzlibrary{angles,intersections,quotes,arrows.meta}

\DeclareMathOperator{\Diag}{Diag}

\DeclareMathOperator{\lcm}{lcm}

\author{Kait Lam \\ \small \texttt{45294583} \\ \small {T02}}
\title{\textsc{Math3306} --- Assignment 4 (Submission)}
\date{25 October 2024}

\def\dotminus{\mathbin{\ooalign{\hss\raise1ex\hbox{.}\hss\cr
  \mathsurround=0pt$-$}}}

\begin{document}

\maketitle


\section*{Question 1}
% \subsection*{Question 1(a)}
\begin{center}
  \textit{This question asks which elements of $(\mathbb N, +)$ and $(\mathbb Z, +)$ are (0-)definable.}
\end{center}
In $(\mathbb N, +)$, we claim that every element is definable.
First, we define a comparison operator by
$x \le y \equiv (\exists z.~ x + z = y)$.
This can be thought of as a syntactic abbreviation if desired.
Now we make an inductive argument.
Starting with zero, we define 0 as the 
least value of the natural numbers,
$\varphi_0(z) \equiv (\forall y.~ z \le y)$.
This exists and is unique by properties of the natural numbers.
For the inductive case, assume 
there exists $\varphi_n$ defining $n$.
Then, we define $n+1$
as $\varphi_{n+1}(n') \equiv \neg \varphi_n(n')  \wedge (\forall y.\,(\exists x. \,\varphi_n(x)\wedge x \le y) \longrightarrow n' \le y)$.
That is, the least number which is greater than (but not equal to) $n$.
Here, 
$(\exists x. \,\varphi_n(x)\wedge x \le y)$
should be thought of as ``$n \le y$''.
This is a funny trick to give a name to the
value of $n$.

In $(\mathbb Z, +)$, only zero is definable.
Zero is the unique value satisfying 
$\varphi_0(z) \equiv (\forall y. \,y + z = y)$ (additive identity).
To see the remaining numbers are undefinable, we should look at the atomic
formulas expressable with this language;
with only a $+$ function, these formulas have the form
$x_1 + \cdots + x_n = y_1 + \cdots + y_m$.
However, whenever such a formula is satisfied by a valuation $\bar b$,
it is also satisfied by $-(\bar b)$---that is, $\bar b$
with its elements negated.
By induction on formula, this extends to all non-atomic formulas in the language.

This means that if a number satisfies a formula with one free variable,
so does its negation.
Therefore, all non-zero numbers cannot be defined by a formula.
Zero may still be definable (since it is equal to its negation),
and indeed we defined it earlier.



\section*{Question 2}
% \subsection*{Question 1(a)}
\begin{center}
  \textit{If $L$ is a language with only relations and $M$ and $N$ are
  elementarily-equivalent finite $L$-structures, then $M$ and $N$ are isomorphic.}
\end{center}
First, assume $L$ has only finitely many relations.
Let $A_M$ and $A_N$ be the object sets of $M$ and $N$, resp.
Consider the atomic diagram $\Diag(M)$.
As $A_M$ is finite and $L$ has finitely many relations,
the set $\Diag(M)$ is also finite.
Let $\varphi$ be the $L(A_M)$-sentence from conjoining all
sentences within $\Diag(M)$.
This is an elementary formula so it also holds in $N_{A_M}$
($N$ extended with constants from $A_M$), and so
$N_{A_M} \models \Diag(M)$.
By Robertson's lemma, this tells us there is an embedding of
$M$ into $N$.
By a symmetric argument, we can start with $\Diag(N)$ to obtain
an embedding of $N$ into $M$.
Since there is an embedding (injective and homomorphic) both ways, the two structures must be isomorphic.

For the general case where $L$ may be infinite, we seek some way to reduce
it to the finite case by exploiting the finiteness of $M$ and $N$.
For the case where the relations in $L$ have bounded arity,
this is easy:
by the boundedness of arity and finiteness of the structures,
there are a finite number of distinct relations within the structure.
Therefore, we proceed as before but we take a finite subset of $\Diag(M)$
with only one representative relation from each equivalence class of equivalent
relations.
We then additionally introduce a (possibly-infinite) number of formulas
of the form $\forall x_1,\ldots,x_n.\, (R(x_1,\ldots,x_n) \longleftrightarrow R'(x_1,\ldots,x_n))$
for each pair of relations in each equivalence class.
These quantified formulas, as well as the formula from a subset of $\Diag(M)$,
are preserved in $N$ by elementary equivalence.
We can use this to show that the full theory of $\Diag(M)$ is modelled by $N$,
then proceed in the same way as the finite case.

When $L$ is infinite and its relations do not have a bounded arity,
we will have to do something else.





\section*{Question 3}
% \subsection*{Question 1(a)}
\begin{center}
  \textit{This question leads us on a journey with existential sentences.}
\end{center}
\subsection*{Question 3(a)}
\begin{center}
  \textit{Existential sentences are preserved by extension.}
\end{center}
\subsection*{Question 3(b)}
\begin{center}
  \textit{Let $M$ and $N$ be models (of what) such that for all existential sentences, $N \models \psi$ implies $M \models \psi$.
  Then, there is a model $L$ such that $L$ is an extension of $N$, and $L$ is elementarily equivalent to $M$.}
\end{center}
\subsection*{Question 3(c)}
\begin{center}
  \textit{Let $T$ be a first-order theory preserved under extensions. Define
    $T_\exists = \{\psi ~|~ \psi\text{ existential and } T \vdash \phi\}$.
    Assume $M \models T_\exists$ and define $\Sigma = \{ \neg \psi ~|~ \psi\text{ existential and }M \nvDash \psi\}$. Then, $T \cup \Sigma$ has a model.
  }
\end{center}
\subsection*{Question 3(d)}
\begin{center}
  \textit{Prove $T_\exists \models T$.}
\end{center}
\subsection*{Question 3(e)}
\begin{center}
  \textit{A sentence is preserved under extensions iff
  it is logically equivalent to an existential sentence.}
\end{center}
Omitted.



\end{document}

