% vim: spelllang=en_au
\documentclass[a4paper]{article}

\usepackage{geometry}
\usepackage{amsmath}
\usepackage{enumerate}
\usepackage{amssymb}

\author{Kait Lam \\ \small \texttt{45294583}}
\title{\textsc{Math3306} --- Assignment 1}
\date{Friday 16 August 2024}

\begin{document}

\maketitle


\section*{Question 1}
\begin{center}
  \textit{This question concerns deriving a FSA to recognise $L^*$, given a FSA to recognise $L$.}
\end{center}

\noindent First, we must (and so, we will)
introduce two language-preserving transformations on FSAs.
% \footnote{
%   That is, the transformation does not change the language recognised by the FSA.
% } transformations on FSAs.
% These will be useful later.
Assume we have a FSA given by $(Q, F, A, \tau, q_0)$ recognising a language $L$.
We introduce:
\begin{itemize}
  \item \textit{Unify accepting states.} It is possible to translate the FSA
    % into a form with a single accepting state and no
    into an equivalent\footnote{recognising the same language} FSA
    which has a single distinguished accepting state, and no outgoing edges from that accepting state.

    This is done by introducing a new symbol $\bullet^+$ with
    $\bullet^+ \notin Q$. We mark this as the only accepting state.
    Finally, for all edges \textit{into} a previously-accepting state, we
    duplicate those to point into $\bullet^+$.

    That is, the new equivalent FSA is
    \begin{align*}
      % (Q states, F final states, A alphabet, tau subset Q x A x Q, Q)
      (Q \cup \{\bullet^+\}, \{\bullet^+\}, A, \tau', q_0)
    % \end{align*}
      \qquad\text{where}~
    % \begin{align*}
      \tau' = \tau \cup
      \big\{(q, a, \bullet^+) \mid \exists q_F \in F.~ (q,a,q_F) \in \tau\big\}.
    \end{align*}
    
    To see that is is equivalent, note that we retain all previous states
    and transition edges, so it behaves identically for transitions which are not terminating.
    For edges handling the last character of a word and going into a
    previously-accepting state, note that the newly-added edges,
    $\tau' \setminus \tau$, will provide the option of moving to the accepting
    $\bullet^+$ state.
    $\bullet^+$ has no outgoing edges, and so will not accept more words than the original language.
    This is, of course, not a deterministic FSA.

  \item \textit{Distinct start state.}
    We can also translate the FSA into a form with a distinguished start state
    which is unreachable aside from the starting configuration.

    We introduce $\bullet^-$ with $\bullet^- \notin Q$ and mark this as the initial state.
    For all outgoing edges from the previous start state, we duplicate these onto $\bullet^-$. Importantly, we do not replicate incoming edges.

    Therefore, the 5-tuple is
    \begin{align*}
      (Q \cup \{\bullet^-\}, F, A, \tau', \bullet^-)
      \qquad\text{where}~
      \tau' = \tau \cup 
      \big\{(\bullet^-, a, q') \mid \exists q' \in Q.~ (q_0, a, q') \in \tau\big\}.
    \end{align*}
    We can see that $\bullet^-$ acts ``{just like}'' the previous $q_0$ for its
    start state behaviour, and the behaviour when $q_0$ is (potentially) revisited is retained by $q_0$ (since the transformation maintains $q_0 \in Q$).
    Thus, we claim it recognises precisely the same language.
\end{itemize}


Now, we are ready to prove the main result.
Suppose we have a FSA (not necessarily deterministic)
recognising $L$.
\begin{enumerate}
  \item 
Apply both the \textit{unify accepting states} and the \textit{distinct start state}
transformations to this FSA (order doesn't matter).

After this, we still have an automaton recognising $L$ but it has distinguished start and accepting states, namely $\bullet^-$ and $\bullet^+$.
As such, all words begin at $\bullet^-$, never revisiting this state,
and all accepted words finish at $\bullet^+$. 
  \item 
    Now for our next trick, we will
    unify our two special states into one state, $\bullet^\pm$.
    That is, we make the substitution of $\bullet^\pm$ in place of
    both $\bullet^-$ and $\bullet^+$ within all components of the 5-tuple representation.

    As some consequences of this, we have $\bullet^\pm$ being the unique initial and accepting state, and this state now has both incoming edges (from $\bullet^+$) and outgoing edges (from $\bullet^-$).

  \item
    We are done.

\end{enumerate}

To make gestures towards why, we consider the words accepted by this automaton.
Since $\bullet^\pm$ is the only accepting state, all accepted words are obtained
from this state.
Moreover, this state may be visited repeatedly, but visits to $\bullet^\pm$
must be interspersed with paths through (something equivalent to) the original $L$ machine, each consuming a word $w \in L$.

As such, the language recognised by this new FSA is the union of words accepted on the $n$-th visit to $\bullet^\pm$, for all $n \in \mathbb N$. This is
\begin{align*}
  \bigcup_{n \in \mathbb N} \big\{ w_1\cdots{}w_n ~\big|~ w_i \in L \big\}
\end{align*}
which is the Kleene star of $L$.

% Informally, we will construct a FSA recognising $L^*$ in the following way.\footnote{
%   This can be simplified through the use of empty ($\epsilon$) edges in the FSA.
% To avoid developing the theory of FSA+$\epsilon$ in this assignment, we will not
% do this here.
% }
% Assume we have a (not necessarily deterministic) FSA recognising $L$, defined by .
% We will perform the following transformations:
%
\section*{Question 2}
\subsection*{Question 2(a)}
\subsection*{Question 2(b)}
\section*{Question 3}
\section*{Question 4}
\subsection*{Question 4(a)}
\subsection*{Question 4(b)}

\end{document}

