\documentclass[a4paper]{article}

\usepackage{geometry}
\usepackage{amsmath}
\usepackage{enumerate}
\usepackage{amssymb}

\author{Kait Lam \\ \small \texttt{45294583}}
\title{\textsc{Math3306} --- Assignment 1}
\date{Friday 16 August 2024}

\begin{document}

\maketitle


\section*{Question 1}
\begin{center}
  \textit{This question concerns deriving a FSA to recognise $L^*$, given a FSA to recognise $L$.}
\end{center}
Informally, we will construct a FSA recognising $L^*$ in the following way.\footnote{
  This can be simplified through the use of empty ($\epsilon$) edges in the FSA.
To avoid developing the theory of FSA+$\epsilon$ in this assignment, we will not
do this here.
}
Assume we have a (not necessarily deterministic) FSA recognising $L$, defined by .
We will perform the following transformations:

\noindent First, we must introduce two semantics-preserving\footnote{
  That is, the transformation does not change the language recognised by the FSA.
} transformations on FSAs.
These will be useful later.
Assume we have a FSA given by $(Q, F, A, \tau, q_0)$ recognising a language $L$.
\begin{itemize}
  \item \textit{Unify accepting states:} It is possible to translate the FSA
    % into a form with a single accepting state and no
    into a semantically equivalent\footnote{recognising the same language} FSA
    which has a single distinguished accepting state, and no outgoing edges from that accepting state.

    This is done by introducing a new symbol $\bullet_+$ with
    $\bullet_+ \notin Q$. We mark this as the only accepting state.
    Finally, for all edges into a previously-accepting state, we
    duplicate those to point into $\bullet_+$.

    That is, the new equivalent FSA is
    \begin{align*}
      (Q \cup \{\bullet_+\}, \{\bullet_+\}, A, \tau', q_0)
    \end{align*}
    where
    \begin{align*}
      \tau' = \tau \cup \big\{(q, a, \bullet_+)
      \mid \exists q_F \in F.~ (q,a,q_F) \in \tau\big\}.
    \end{align*}




\end{itemize}

\section*{Question 2}
\subsection*{Question 2(a)}
\subsection*{Question 2(b)}
\section*{Question 3}
\section*{Question 4}
\subsection*{Question 4(a)}
\subsection*{Question 4(b)}

\end{document}

