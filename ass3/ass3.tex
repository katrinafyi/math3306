% vim: spelllang=en_au
\documentclass[a4paper]{article}

\usepackage{geometry}
\usepackage{amsmath}
% \usepackage{blkarray}
\usepackage{enumerate}
\usepackage{amssymb}
\usepackage{amsthm}
\usepackage{hyperref}
\usepackage{minted}
% \usepackage[symbol]{footmisc}
\usepackage{tikz}
\usetikzlibrary{automata}
\usetikzlibrary{positioning,arrows.meta,calc}
\usetikzlibrary{angles,intersections,quotes,arrows.meta}


\DeclareMathOperator{\lcm}{lcm}

\author{Kait Lam \\ \small \texttt{45294583} \\ \small {T02}}
\title{\textsc{Math3306} --- Assignment 3 (Submission)}
\date{4 October 2024}

\def\dotminus{\mathbin{\ooalign{\hss\raise1ex\hbox{.}\hss\cr
  \mathsurround=0pt$-$}}}

\begin{document}

\maketitle


\section*{Question 1}
% \subsection*{Question 1(a)}
\begin{center}
  \textit{Prove the order-isomorphism between a well-ordered set and its set of segments.}
\end{center}
Let $(X, \le)$ be woset, and let $A = \{X_a ~|~ a \in X\}$ be its set of segments.
Recall that a segment is $X_a = \{ x \in X ~|~ x < a \}$.
We will prove an order-isomorphism between $(X, \le)$ and $(A, \subseteq)$.
Define $f : X \to A$ by $a \mapsto X_a$.
Now, we check the properties of injectivity, surjectivity, and order-preservation:
\begin{itemize}
  \item 
    Injectivity: Let $a, b \in X$ and $a \ne b$. Without loss of generality,
    assume $a < b$ (by total ordering).
    Then, $a \in X_b$ and $a \notin X_a$ and so $X_a \subset X_b$. 
    Thus,
    $f(a) = X_a \subset X_b = f(b)$ and $f(a) \ne f(b)$.

  \item Surjectivity:
    Let $x \in A$ be arbitrary. By definition of $A$, there must exist $a \in X$ such that $X_a = x$.
    Then, $f(a) = X_a = x$.

  \item Order-preserving:
    Let $a, b \in X$ and $a < b$.
    Then, we have that $x < a \implies x < b$ for all $x$,
    and so the segments $X_a$ and $X_b$ are subsets (or equal).
    We then have $X_a \subsetneq X_b$ since $a \in X_b$ but $a \notin X_a$.
\end{itemize}

\section*{Question 2}
\begin{center}
  \textit{Consider $(\mathbb N, <)$
  as a model in LAST and check the validity of some axioms of ZFC.}
\end{center}
In a world where $\in$ is $<$ and the universe is $\mathbb N$, \ldots
\begin{enumerate}[a)]
  \item \textit{Ext} holds. Suppose not, then there exists $x,y \in \mathbb N$
    such that $\forall z(z < x \longleftrightarrow z < y)$
    and $x \ne y$. Without loss of generality, assume $x < y$.
    By the bi-implication, this implies $x < x$ which is not valid in the model.
  \item \textit{Un} holds---that is, given a $x$, there exists $u$ such that
    $\forall z(z \in u \longleftrightarrow \exists y(y \in x \wedge z \in y))$.
    For a given $x$, choose $u = x \dotminus 1$ (clamped subtraction, returning 0 if the second number is larger than the first).

    To show that this satisfies the axiom, note that,
    in $(\mathbb N,<)$,
    the formula
    $\exists y (y \in x \wedge z \in y)$ is equivalent to $\exists y(z < y < x)$
    which is equivalent to $z + 1 < x$.
    Thus, the required property of $u$ becomes
    $\forall z(z < u \longleftrightarrow z + 1 < x)$.
    This clearly holds when $u + 1 = x$ (whenever $x \ne 0$) as then,
    $z+1 < x \iff z + 1 < u + 1 \iff z < u$.
    When $x = 0$, we have $u = 0 \dotminus 1 = 0$
    and both sides of the $``\longleftrightarrow"$
    are never satisfied, 
    since no number is less than zero, making them equivalent.

  \item \textit{Pow} also holds.
    For a given $x$, choose $p = x + 1$.
    Then, for all $z$,
    \begin{align*}
      z < p \longleftrightarrow z < x + 1 &\longleftrightarrow z \le x \\
            &\longleftrightarrow \forall a(a < z \longrightarrow a < x) \\
            &\longleftrightarrow \forall a(a \in z \longrightarrow a \in x) \\
            &\longleftrightarrow z \subseteq x
    \end{align*}
    which gives us the axiom of power-set,
    $\forall x\exists p\forall z(z \in p \longleftrightarrow z \subseteq x)$.


  \item \textit{Fnd} holds.
    First note that we interpret the empty set as the numeral zero (since nothing is less than zero, just like nothing is within the empty set).
    For any non-zero $x$, we can choose $y = 0$ which is both less than $x$ and
    there does not exist any numbers which are both less than $y = 0$
    and less than $x$.
  \item \textit{Inf} does not hold. The axiom is written as
    $\exists w(\emptyset \in w \wedge \forall x(x \in w \longrightarrow (x \cup \{x\} \in w)))$.
    To sidestep the possibly-problematic definitions of singleton, we expand the abbreviation with
    \begin{align*}
      x \in w \longrightarrow x \cup \{x\} \in w 
      ~\iff~
      x \in w \longrightarrow\exists y\left[y \in w \wedge \forall a(a \in y \longleftrightarrow (a \in x \vee a = x))\right].
    \end{align*}
    Suppose the axiom held and there was such a $w$. To derive a contradiction,
    we choose $x = w \dotminus 1 < w$ (noting $w \ne 0$).
    By consequence of the expanded axiom, we can obtain a $y$
    such that $y < w$ and for all $a$,
    \begin{align*}
    a < y \longleftrightarrow (a < w \dotminus 1 \vee a = w \dotminus 1) 
    ~\iff~
    a < y \longleftrightarrow a < w.
    \end{align*}
    By extensionality (which holds), this implies $y = w$ which contradicts
    the property $y < w$ (required by $y \in w$ within the expanded axiom).
\end{enumerate}


\section*{Question 3}
\begin{flushleft}
  \textit{Let $A$ be a set of ordinals. Prove that exactly one of the following holds:}
  \begin{enumerate}[(i)]
    \item \textit{$\bigcup A$ is the largest element of $A$, or}
    \item \textit{$A$ has no largest element, $\bigcup A \notin A$, and 
      $\bigcup A$ is not the successor of any other ordinal.}
  \end{enumerate}
  \textit{Conclude that a non-zero ordinal $\lambda$ is a limit ordinal if and only if
  $\bigcup \lambda = \lambda$.}
\end{flushleft}
In these proofs, keep in mind
that the big union of a set of ordinals is an ordinal.
We will prove this in two parts:
\begin{itemize}
  \item $\text{(ii)} \implies \neg\text{(i)}$:
    This is apparent. (ii) states that $A$ has no largest
    element and $\bigcup A \notin A$.
  \item $\neg\text{(ii)} \implies \text{(i)}$:
    After applying De Morgan's laws to $\neg$(ii), we have three cases from each of the terms within (ii).
    Note that these are negated, since $\neg(p \wedge q) \equiv \neg p \vee \neg q$.
    \begin{itemize}
      \item $A$ has a largest element$ \implies $(i):
        Suppose $A$ has a largest element, $\alpha \in A$,
        such that for all $\alpha' \in A$ where $\alpha' \ne \alpha$,
        we have $\alpha' < \alpha$.
        As $\alpha$ and $\alpha'$ are both ordinals, 
        this also means $\alpha' \subset \alpha$.

        For any sets $x,y$, we know that if $x \subseteq y$ then $x \cup y = y$,
        and this also applies to the big union.
        Therefore, $\bigcup A$ is precisely the largest element of $A$.

      \item $\bigcup A \in A \implies $(i):
        Let $\alpha \in A$ be arbitrary. Since the big union of $A$ contains 
        all elements of $A$ as subsets,
        we have $\alpha \subseteq \bigcup A$ and $\alpha \le \bigcup A$.
        And so $\bigcup A$ is the largest element.
      \item $\bigcup A$ is the successor of some other ordinal$ \implies $(i):
        Let $\alpha$ be an ordinal such that $\bigcup A = (\alpha \cup \{\alpha\})$.
        Because $\alpha \in \bigcup A$, there exists $\beta \in A$ such that
        $\alpha \in \beta \in A$.
        We will show that $\beta = \bigcup A$.

        To see $\bigcup A \subseteq \beta$, note that $\alpha \in \beta$
        and $\alpha \subset \beta$ (as $\alpha$ and $\beta$ are ordinals).
        Then, the union of $\{\alpha\}$ and $\alpha$, namely $\bigcup A$, must be
        contained within $\beta$.
        For $\beta \subseteq \bigcup A$, this is by a simple property of unions.
        As $\beta \in A$, we must have $\beta \subseteq \bigcup A$.

        This gives us $\bigcup A \in A$ and we can apply the previous case
        to conclude (i). 
    \end{itemize}
\end{itemize}
Now, we use this result to show a non-zero ordinal $\lambda$ is a limit ordinal
if and only if $\bigcup \lambda = \lambda$.
\begin{itemize}
  \item $(\Longleftarrow)$:
    If $\bigcup \lambda = \lambda$, then it cannot be the case that $\bigcup \lambda = \lambda \in \lambda$ --- that would violate the axiom of foundation.
    In particular, this means that $\bigcup \lambda$ is not the greatest element of $A$.
    This
    is in conflict with case (i) of the lemma,
    leaving us in case (ii).
    Since $\lambda$ 
    is not a successor (due to (ii)) and it is not zero,
    it must be a limit ordinal.
  \item $(\Longrightarrow)$:
    Assume $\lambda$ is not a successor ordinal and is not zero.

    To show $\lambda \subseteq \bigcup \lambda$,
    we begin with an arbitrary $\alpha \in \lambda$.
    We have $a \subseteq \bigcup \lambda$
    by a property of the big union.
    We show that we cannot have $\alpha =  \bigcup \lambda$. 
    If this was the case, then we would have $\bigcup \lambda \in \lambda$
    and by the ``$\bigcup A \in A \implies $(i)'' case of the lemma,
    this would imply that $\bigcup \lambda$ is the largest element.
    From this, we could write
    $\lambda = (\,\bigcup \lambda\,) \cup \{\bigcup \lambda\}$,
    since $\bigcup \lambda$ is the largest ordinal less than $\lambda$.
    This contradicts the assumption $\lambda$ is a limit ordinal.
    Therefore, $\alpha \subset \bigcup \lambda$ and so $\alpha \in \bigcup \lambda$, and
    we obtain $\lambda \subseteq \bigcup \lambda$.

    For $\bigcup \lambda \subseteq \lambda$,
    choose $\alpha \in \bigcup \lambda$.
    Then, there exists $\beta \in \lambda$ such that $\alpha \in \beta \in \lambda$.
    As $\beta$ and $\lambda$ are both ordinals,
    $\alpha \in \beta \subset \lambda$ and so $\alpha \in \lambda$
    and the non-strict subset holds.








\end{itemize}



\section*{Question 4}
\begin{center}
  \textit{Given $\alpha \ne 0$ an ordinal and $\lambda$ a limit ordinal,
  prove that $\alpha \cdot \lambda$ is a limit ordinal.}
\end{center}
First, note that $\alpha \cdot \lambda \ne 0$ since
$\alpha \ne 0$ and $\omega \le \lambda$, and so
$\alpha \cdot 0 \in \alpha \cdot 1 \subseteq \bigcup_{\delta < \lambda}(\alpha \cdot \delta)=\alpha \cdot \lambda$.

We now prove the main statement by contradiction.
Suppose $\alpha \cdot \lambda$ is a successor ordinal.
Then, there exists $\Lambda$, an ordinal, such that
$\alpha \cdot \lambda = \Lambda \cup \{\Lambda\}$.
Using the recursive definition of multiplication,
$\alpha \cdot \lambda = \bigcup_{\delta < \lambda}(\alpha \cdot \delta) = \Lambda \cup \{\Lambda\}$.
Note that this implies $\Lambda$ is the greatest element of $\alpha \cdot \lambda$.
Since $\Lambda \in \bigcup_{\delta < \lambda}(\alpha \cdot \delta)$,
there exists $\gamma \in \lambda$
such that $\Lambda \in \alpha \cdot \gamma$.

We now show that this implies
that $\gamma$ is the maximum of $\lambda$ (a contradiction).
To do this, we use contradiction again (I'm sorry);
suppose there exists $\gamma' \in \lambda$ and $\gamma < \gamma'$.
Then, we would have $\alpha \cdot \gamma < \alpha \cdot \gamma'$ (see lemma below)
and so there exists $\Lambda'$ such that $\Lambda' \in \alpha \cdot \gamma'$ but
$\Lambda' \notin \alpha \cdot \gamma$
(so, by total ordering of ordinals, $\Lambda' \subseteq \alpha \cdot \gamma$).
However, we would still have
$\alpha \cdot \gamma' \subseteq \bigcup_{\delta < \lambda}(\alpha \cdot \delta)=\alpha \cdot \lambda$.
Therefore,
$\Lambda \in \alpha \cdot \gamma ~\subseteq~ \Lambda' \in \alpha \cdot \gamma' ~\subseteq~ \alpha \cdot \lambda$
and $\Lambda'$ is a larger element than $\Lambda$, contradicting the maximality
of $\Lambda$.

Concluding, we have found a maximum of $\lambda$
due to $\alpha \cdot \lambda$ having a maximum element.
This contradicts the assumption $\lambda$ was a limit ordinal,
and so $\alpha \cdot \lambda$ mustn't
be a successor ordinal.
We conclude $\alpha \cdot \lambda$ is a limit ordinal.

% \vspace{0.7em}
% \noindent
Within this proof, we used a little lemma which we will show now---namely, that $\alpha < \beta \implies \gamma \cdot \alpha < \gamma \cdot \beta$,
for non-zero $\gamma$ and any ordinals $\alpha, \beta$
(i.e.\ ordinal multiplication
is strictly increasing in its second argument).
The proof is by induction and cases on $\beta$.
If $\beta = 0$, the lemma is vacuously true.
Otherwise,
if $\beta$ is a successor ordinal,
$\gamma \cdot \beta = \gamma \cdot \operatorname{pred}(\beta) + \gamma$.
If $\alpha = \operatorname{pred}(\beta)$, then the result is apparent
using ``$\forall \text{ ordinal }\alpha,\beta : \alpha \ne 0 \implies \beta < \beta + \alpha$''.
If $\alpha < \operatorname{pred}(\beta)$,
we apply the inductive hypothesis.
Finally,
if $\beta$ is a limit ordinal, 
then
 $\alpha + 1 \in \beta$ since $\alpha \in \beta$.
 % and $\gamma \cdot (\alpha + 1) \subseteq \gamma \cdot \beta$.
Therefore, $\gamma \cdot \alpha ~<~ (\gamma \cdot \alpha + \gamma) ~=~ \gamma \cdot (\alpha + 1) ~\subseteq~ =~ \bigcup_{\delta < \beta}(\gamma \cdot \delta) ~=~\gamma \cdot \beta$.
(The leftmost inequality is due to right-monotonicity of $+$.)
This concludes the proof.


% Here, we use the definition of
% $\gamma \cdot\alpha = \sum_{\xi < \alpha}\gamma=\operatorname{Ord}(A_\alpha, <_{A_\alpha})$
% where $A_\alpha = \bigcup_{\xi < \alpha} (\gamma \times \{\xi\})$
% and similarly for $\gamma \cdot \beta$.
% We see that $A_\alpha$ is a strict subset of $A_\beta$ and so
% their corresponding ordinals will be ordered in the same manner.






\end{document}

