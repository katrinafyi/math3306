% vim: spelllang=en_au
\documentclass[a4paper]{article}

\usepackage{geometry}
\usepackage{amsmath}
\usepackage{blkarray}
\usepackage{enumerate}
\usepackage{amssymb}
\usepackage{amsthm}
\usepackage{hyperref}
\usepackage{minted}
% \usepackage[symbol]{footmisc}
\usepackage{tikz}
\usetikzlibrary{automata}
\usetikzlibrary{positioning,arrows.meta,calc}
\usetikzlibrary{angles,intersections,quotes,arrows.meta}


\DeclareMathOperator{\lcm}{lcm}

\author{Kait Lam \\ \small \texttt{45294583} \\ \small {T02}}
\title{\textsc{Math3306} --- Assignment 3 (Submission)}
\date{4 October 2024}

\def\dotminus{\mathbin{\ooalign{\hss\raise1ex\hbox{.}\hss\cr
  \mathsurround=0pt$-$}}}

\begin{document}

\maketitle


\section*{Question 1}
% \subsection*{Question 1(a)}
\begin{center}
  \textit{Prove the order-isomorphism between a well-ordered set and its set of segments.}
\end{center}
Let $(X, \le)$ be woset, and let $A = \{X_a ~|~ a \in X\}$ be its set of segments.
We will prove an order-isomorphism between $(X, \le)$ and $(A, \subseteq)$.
Define $f : X \to A$ by $a \mapsto X_a$.
Now, we check the properties of injectivity, surjectivity, and order-preservation:
\begin{itemize}
  \item 
    Injectivity: Let $a, b \in X$ and $a \ne b$. Without loss of generality,
    assume $a < b$ (by total ordering).
    Then, $a \in X_b$ and $a \notin X_a$ and so $X_a \subset X_b$. 
    Thus,
    $f(a) = X_a \subset X_b = f(b)$ and $f(a) \ne f(b)$.

  \item Surjectivity:
    Let $x \in A$ be arbitrary. By definition of $A$, there must exist $a \in X$ such that $X_a = x$.
    Then, $f(a) = X_a = x$.

  \item Order-preserving:
    Let $a, b \in X$ and $a < b$.
    Then, we have that $x < a \implies x < b$ for all $x$,
    and so the segments $X_a$ and $X_b$ are subsets (or equal).
    We then have $X_a \subsetneq X_b$ since $a \in X_b$ but $a \notin X_a$.
\end{itemize}

\section*{Question 2}
\begin{center}
  \textit{Consider $(\mathbb N, <)$
  as a model in LAST and check the validity of some axioms of ZFC.}
\end{center}
In a world where $\in$ is $<$ and the universe is $\mathbb N$, \ldots
\begin{enumerate}[a)]
  \item \textit{Ext} holds. Suppose not, then there exists $x,y \in \mathbb N$
    such that $\forall z(z < x \longleftrightarrow z < y)$
    and $x \ne y$. Without loss of generality, assume $x < y$.
    By the bi-implication, this implies $x < x$ which is not valid in the model.
  \item \textit{Un} holds---that is, given a $x$, there exists $u$ such that
    $\forall z(z \in u \longleftrightarrow \exists y(y \in x \wedge z \in y))$.
    For a given $x$, choose $u = x \dotminus 1$ (clamped subtraction, returning 0 if the second number is larger than the first).

    To show that this satisfies the axiom, note that,
    in $(\mathbb N,<)$,
    the formula
    $\exists y (y \in x \wedge z \in y)$ is equivalent to $\exists y(z < y < x)$
    which is equivalent to $z + 1 < x$.
    Thus, the required property of $u$ becomes
    $\forall z(z < u \longleftrightarrow z + 1 < x)$.
    This clearly holds when $u + 1 = x$ (whenever $x \ne 0$) as then,
    $z+1 < x \iff z + 1 < u + 1 \iff z < u$.
    When $x = 0$, we have $u = 0 \dotminus 1 = 0$
    and both sides of the $``\longleftrightarrow"$
    are never satisfied, 
    since no number is less than zero, making them equivalent.

  \item \textit{Pow} also holds.
    For a given $x$, choose $p = x + 1$.
    Then, for all $z$,
    \begin{align*}
      z < p \longleftrightarrow z < x + 1 &\longleftrightarrow z \le x \\
            &\longleftrightarrow \forall a(a < z \longrightarrow a < x) \\
            &\longleftrightarrow \forall a(a \in z \longrightarrow a \in x) \\
            &\longleftrightarrow z \subseteq x
    \end{align*}
    which gives us the axiom of power-set,
    $\forall x\exists p\forall z(z \in p \longleftrightarrow z \subseteq x)$.


  \item \textit{Fnd} holds.
    First note that we interpret the empty set as the numeral zero (since nothing is less than zero, just like nothing is within the empty set).
    For any non-zero $x$, we can choose $y = 0$ which is both less than $x$ and
    there does not exist any numbers which are both less than $y = 0$
    and less than $x$.
  \item \textit{Inf} does not hold. The axiom is written as
    $\exists w(\emptyset \in w \wedge \forall x(x \in w \longrightarrow (x \cup \{x\} \in w)))$.
    To sidestep the possibly-problematic definitions of singleton, we expand the abbreviation with
    \begin{align*}
      x \in w \longrightarrow x \cup \{x\} \in w 
      ~\iff~
      x \in w \longrightarrow\exists y\left[y \in w \wedge \forall a(a \in y \longleftrightarrow (a \in x \vee a = x))\right].
    \end{align*}
    Suppose the axiom held and there was such a $w$. To derive a contradiction,
    we choose $x = w \dotminus 1 < w$ (noting $w \ne 0$).
    By consequence of the expanded axiom, we can obtain a $y$
    such that $y < w$ and for all $a$,
    \begin{align*}
    a < y \longleftrightarrow (a < w \dotminus 1 \vee a = w \dotminus 1) 
    ~\iff~
    a < y \longleftrightarrow a < w.
    \end{align*}
    By extensionality (which holds), this implies $y = w$ which contradicts
    the property $y < w$ within the axiom.
\end{enumerate}


\section*{Question 3}
  \textit{Let $A$ be a set of ordinals. Prove that exactly one of the following holds:}
  \begin{enumerate}[(i)]
    \item \textit{$\bigcup A$ is the largest element of $A$, or}
    \item \textit{$A$ has no largest element, $\bigcup A \notin A$, and 
      $\bigcup A$ is not the successor of any other ordinal.}
  \end{enumerate}
  \textit{Conclude that a non-zero ordinal $\lambda$ is a limit ordinal if and only if
  $\bigcup \lambda = \lambda$.}



\section*{Question 4}
\begin{center}
  \textit{Given $\alpha \ne 0$ an ordinal and $\lambda$ a limit ordinal,
  prove that $a \cdot \lambda$ is a limit ordinal.}
\end{center}



\end{document}

